Cette partie explique comment installer les 3 parties de la solution Vigilate dans un environnement de production. Les environnements de production et de développement sont quasis identique pour le backend et le scanner de programme, car ils sont codé en python et ont juste besoin d’un éditeur de texte pour être développé. Le frontend, quant à lui, nécessite des outils supplémentaire pour l’environnement de développement. Pour le backend il faut essentiellement python, django, nginx, un gestionnaire de bases de données, quelques modules pythons, et uwsgi qui fera la liaison entre nginx et Django. Pour le scanner juste python3 et le modules requests. En ce qui concerne le front end il va falloir installer nodejs et npm, un gestionnaire de paquets pour Javascript.
Linux est indispensable pour la partie serveur. Il faut ensuite vérifier que Python 3 est à jour. Le serveur est en Django il faut donc l’installer ainsi que les autres modules python nécessaires. Pour finir l’installation, il faut ensuite cloner le projet depuis github. Une série de commandes décrite plus bas, est utile pour le frontend en nginx.
Pour installer les dépendances vers le scanner de programme, il y a 3 méthodes différentes selon l’os: mac, linux ou windows.