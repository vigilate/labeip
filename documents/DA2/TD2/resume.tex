Il est nécessaire d’avoir une documentation technique pour aider les développeurs à comprendre correctement l’organisation du projet, que ce soit l’architecture ou la façon de travailler.
Tout d’abord Vigilate est divisé en 4 grands axes : le site web, le programme de scan, le backend, la machine virtuelle.
La norme utilisée est Pep8, c’est en effet la norme recommandée pour le développement Python.
Nous avons mis en place une série de tests sur Travis, qui permet de faire les tests en continu à chaque ajout de fonctionnalité. Avant chaque release, nous faisons passer des tests de sécurité pour garantir à nos utilisateurs un outil dont ils peuvent avoir confiance.
Le projet est mis en ligne sur github, nous avons mis en place des tags pour chaque nouvelle release. 