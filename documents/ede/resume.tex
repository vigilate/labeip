Ce document détaille l'étude de l'existant de notre EIP Vigilate. L'EDE commence par un rappel de notre sujet d'EIP, Vigilate, un outil de sécurité informatique permettant d'informer rapidement les utilisateurs des nouvelles vulnérabilités connues qui les concernent.\\
Différentes solutions existent sur le marché, nous allons les aborder dans ce document.\\
Nesus est l’un des plus importants scanner de vulnérabilités, mais il va devoir scanner de nombreux plugins avant de tester celui qui nous intéresse réellement.\\
Shalvik Protect quant à lui, va servir à installer les corrections proposées par les éditeurs des programmes vulnérables mais ne peut que détecter l’absence de ces correctifs et pas la réelle présence d’une vulnérabilité.\\
Le CERT/CSIRT n’est pas un programme, mais un groupe de personnes dédié à la sécurité dans une entreprise, il va prendre en compte toutes les étapes pour avoir un système d’information sécurisé.\\
Cisco Security Manager est un SIEM (Security Event Information Management) qui permet d’afficher en temps réel les évènements de sécurité de l'infrastructure réseau, cependant un certain niveau de connaissance réseau est nécessaire pour pouvoir utiliser cet outil.\\
CERT-XMCO est ce qui se rapproche le plus de Vigilate, néanmoins le filtrage n’est pas très précis et le service n’est pas accessible publiquement.\\
Dans la 3eme partie du document nous montrerons comment notre solution se distingue des autres au niveau du fonctionnement: il ne s’agit pas d’un véritable scanner de vulnérabilité mais d’un service qui permet d’avertir rapidement les utilisateurs qui souhaitent suivre l’état de vulnérabilité de certains programmes particuliers.\\
En conclusion, une synthèse des différences entre les solutions montre que Vigilate se positionne dans un endroit qui n’est pas assez couvert dans le marché actuel des solutions de suivis de vulnérabilités.\\