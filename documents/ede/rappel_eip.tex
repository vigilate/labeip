\section{Qu'est-ce qu’un EIP et Epitech}
Epitech,  école de l'innovation et de l'expertise informatique propose un cursus en 5 années, basé sur une pédagogie par projet (à réaliser seul ou en groupe).  L'un de ces projets, l’EIP (Epitech Innovating Project), réalisé par groupe de 6 à 15 étudiants, démarre au cours de la 3ème année et se déroule sur 2 ans et demi. Ce  projet doit être particulièrement innovant car son objectif est d’être commercialisable à la fin de la 5ème année d’Epitech.

\section{Sujet de votre EIP}
Le but de Vigilate est d’avertir les utilisateurs des services obsolètes ou potentiellement vulnérables affectant en particulier leur infrastructure (sites web, réseau d'entreprise, logiciels), dans le but de les informer des risques techniques encourus et des éventuelles mises à jour ou corrections à appliquer.
Cela sans effectuer de scan de vulnérabilité.
On proposons cependant un outil de scan de programme qui envoie la liste sur notre plateforme web.

\section{Glossaire}
\noindent
- C -\\
CVE (Common Vulnerabilities and Exposures): Numéro qui renvoie à la fiche descriptive complète d'une vulnérabilité précise. Exemple: CVE-2013-4343.\\
CSIRT (Computer Security Incident Response Team): Équipe chargée d'assurer la sécurité d'une entreprise.\\
CERT (Computer Emergency Response Team): marque déposée représentant les CSIRT certifiés.\\
- F -\\
Fingerprint: Déduction de la version d'un programme/système d'exploitation en fonction de la façon dont il répond à nos requêtes sur le réseau.\\
Fork: Création d'un nouveau programme en utilisant le code source du premier.\\
- G -\\
GPL (General Public License): Licence dédié aux logiciels libres.\\
- P -\\
Patch: Un correctif qui corrige un bug ou une vulnérabilité dans un programme.\\
- S -\\
SCADA (Supervisory Control and Data Acquisition): Système de télégestion à grande échelle utilisé dans l'industrie.\\

