\section{CERT/CSIRT}
\thispagestyle{plain}
\subsection{Présentation}
Les CSIRT (Computer Security Incident Response Team) ou CERT (Computer Emergency Response Team) sont des équipes dédié à la gestion de la sécurité informatique et résoudre les incidents qui en sont lié.

\subsection{Historique}
Le premier CERT (nomé CERT/CC)  a été crée en 1988 dans l’université américaine de Carnegie Mellon suite au premier vers informatique (Moris Worm). [1]
Au début composé d’une petite équipe, ce CERT compte actuellement 150 professionnelles de la sécurité informatique. [2]\\

CERT étant une marque déposé par CMU (Carnegie Mellon University) [3], le terme CSIRT est aussi utilisé.\\

Les CSIRT souhaitant utiliser le terme CERT peuvent en faire la demande auprès de CMU.\\
Aujourd’hui, il existe en France 17 CSIRT validé par le CMU. [4]\\

\subsection{Description}
Les CSIRT ont plusieurs rôles:
\begin{itemize}
\item [$\bullet$] Centralisation d’information suite à un attaque informatique\\
\subitem [$\bullet$] Réception des information\\
\subitem [$\bullet$] Analyse et corrélation des incidents\\
\item [$\bullet$] Traitement des alertes et réaction aux attaques\\
\subitem [$\bullet$] Analyse technique\\
\subitem [$\bullet$] Partage d’information avec les autres CSIRT\\
\subitem [$\bullet$] Intervention d’urgence\\
\item [$\bullet$] Prévention/Formation\\
\subitem [$\bullet$] Diffusion d’information dans le but de minimiser le risque d’incident et leurs éventuels conséquences.\\
\end{itemize}
\subsection{Points positifs}
Gestion de bout en bout de la sécurité informatique du client.

\subsection{Points Négatifs}
Dans certains cas, équipe dédié entièrement à un seul client.\\
Non adapté à une utilisation hors entreprise.\\
Nécessite d’avoir une personne physique dédié pour gérer la sécurité.\\

\subsection{Références}
\small
\noindent
[1][2] \url{http://www.cert.org/about}\newline
[3][4] \url{http://www.cert.org/incident-management/csirt-development/cert-authorized.cfm}\newline
\normalsize