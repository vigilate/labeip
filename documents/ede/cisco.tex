\section{Cisco Security Manager}
\thispagestyle{plain}
\subsection{Présentation}
Cisco Security Manager permet aux organisations de gérer leurs infrastructures Cisco de manière centralisée et surveiller les menaces de sécurité potentielles.

\subsection{Historique}
Cisco Systems est une entreprise informatique américaine spécialisée, à l’origine, dans le matériel réseau (routeur et commutateur ethernet), et depuis 2009 dans les serveurs. Elle a été fondée en 1984 par Leonard Bosack et Sandra Lerner à San Francisco.

\subsection{Description}
Cette solution se présente sous la forme d'un programme à installer sur un ordinateur. Son interface agrège en temps réel les évènements récupérés sur les différents équipements réseau. Plusieurs types d'affichages sont possibles : liste exhaustive, liste des évènements les plus graves sur les X dernières heures, graphique de répartition des alertes, etc.\\
C'est une interface très complète qui recense un grand nombre d'options.

\subsection{Points positifs}
Réutilisation des objets et des règles de sécurité pour surveiller les menaces qui pèsent sur la sécurité et minimiser les erreurs potentielles.\\
Outils intégrés de bout en bout pour faciliter l'application cohérente de la politique et le dépannage rapide.\\
Gestion consolidée des évènements pour permettre la consultation des évènements historiques et en temps réel.\\

\subsection{Points négatifs}
Nécessite un utilisateur formé aux bases de la sécurité et du réseau.
La solution ne se destine qu’à certaines gammes de produits Cisco.

\subsection{Références}
\small
\noindent
\url{http://www.cisco.com/web/FR/products/security/security_manager.html}
\normalsize