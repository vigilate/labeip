\section{Matrice de préférence}
\thispagestyle{plain}

Dans cette matrice de préférence, nous comparons plusieurs points importants. La réactivité à une nouvelle menace, la simplicité d’utilisation de l’outil, qui est ciblé par cet outil et enfin la pertinence de l’information qui est remontée.\\

\small
\begin{tabular}{|p{2.5cm}|p{3.5cm}|p{3.5cm}|p{2cm}|p{3.5cm}|}
  \hline
  \rowcolor{Gainsboro}Projet & Réactivité & Simplicité & Cible & Pertinence de l'information \\
\hline

  Nessus & Plutôt longue. Réactivité éditeur + temps de scan & Les scans classiques sont simplistes à utiliser mais la personnalisation de ceux-ci est plus compliqué & Entreprises. & Bonne. Seules les vulnérabilités trouvées sont remontées. \\
\hline

  Shalvik Protect & Réactivité égale à celle des éditeurs de programmes. & Très simple il scan et installe les patchs tout seul. & Entreprises. & Bonne. Elle concerne seulement les programmes installés sur la machine. \\
\hline

  CERT/CSIRT & Très bonne. L’équipe est prête à répondre en cas d’urgence. & Très simple pour l’entreprise, ce n'est pas elle qui s’en occupe. & Entreprises, organisations. & Bonne. Elle est adaptée à la cible.\\
\hline

  Cisco Security Manager & Bonne, dès qu’un événement interne est détecté. & Difficile. Nécessite une connaissance des technologies réseaux. & Entreprises, organisations & Bonne, elle concerne des événements internes.\\
\hline

  CERT-XMCO & Bonne. Alerte dès qu’un bulletin est publié. & Assez simple. & Entreprises, organisations. & Mauvaise. Touts les bulletins sont affichés, et le filtrage ne se fait pas forcément très bien. \\
\hline

  Vigilate & Très bonne. Dès qu’une vulnérabilité est rendu publique. & Très simple. Ergonomique. & Entreprises, organisations, revendeurs, webmasters, particuliers & Très bonne. Seule l’information qui concerne l’utilisateur est remonté. \\
\hline
\end{tabular}
\normalsize  


\section{SWOT}
\begin{tabular}{|l|p{5cm}|p{5cm}|}
  \hline
  \rowcolor{Gainsboro} & Positive & Négative \\
  \hline

  Interne & \textbf{Forces}: Pertinence des informations obtenues (personnalisation) Rapidité de l’information & \textbf{Faiblesses}: Pas de scanner du vulnérabilité
  \\
  \hline

  Externe & \textbf{Opportunités}: Besoin de plus de sécurité & \textbf{Menaces}: Beaucoup d’outils qui existent déjà \\
  \hline
\end{tabular}
