\section{Matrice de préférence}
\thispagestyle{plain}

Dans cette matrice de de préférence, nous comparons plusieurs points important. La réactivité à une nouvelle menace, la simplicité d’utilisation de l’outil, qui est ciblé par cet outil et enfin la pertinence de l’information qui est remonté.\\

\small
\begin{tabular}{|p{2.5cm}|p{3.5cm}|p{3.5cm}|p{2cm}|p{3.5cm}|}
  \hline
  \rowcolor{Gainsboro}Projet & Réactivité & Simplicité & Cible & Pertinence de l'information \\
\hline

  Nessus & Plutôt longue. Réactivité éditeur + temps de scan & Les scans classiques sont simpliste à utiliser mais la personnalisation de ceux-ci est plus compliqué & Entreprise. & Bonne. Seul les vulnérabilité trouvé sont remontés. \\
\hline

  Shalvik Protect & Réactivité égale à celle des éditeur de programmes. & Très simple il scan et installe les patchs tout seul. & Entreprises. & Bonne. Elle concerne seulement les programmes installé sur la machine. \\
\hline

  CERT/CSIRT & Très bonne. L’équipe est prête à répondre en cas d’urgence. & Très simple pour l’entreprise, c’est pas elle qui s’en occupe. & Entreprises, organisation. & Bonne. Elle est adapté à la cible.\\
\hline

  Cisco Security Manager & Bonne, dès qu’un évènement interne est détecté. & Difficile. Nécessite une connaissance des technologies réseaux. & Entreprises, organisations & Bonne, elle concerne des évènements interne.\\
\hline

  CERT-XMCO & Bonne. Alerte dès qu’un bulletin est publié. & Assez simple. & Entreprise, organisation. & Mauvaise. Touts les bulletins sont affiché, et le filtrage ne se fait pas forcément très bien. \\
\hline

  Vigilate & Très bonne. Dès qu’une vulnérabilité est rendu publique. & Très simple. Ergonomique. & Entreprise, organisation, revendeur, webmaster, particulier & Très bonne. Seul l’information qui concerne l’utilisateur est remonté. \\
\hline
\end{tabular}
\normalsize  


\section{SWOT}
\begin{tabular}{|l|p{5cm}|p{5cm}|}
  \hline
  \rowcolor{Gainsboro} & Positive & Négative \\
  \hline

  Interne & Forces & Faiblesses \\
  \hline

  Externe & Opportunités & Menaces \\
  \hline
\end{tabular}
