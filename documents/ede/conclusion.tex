\section{Matrice de préférence}
\thispagestyle{plain}

Dans cette matrice de préférence, nous comparons plusieurs points importants: la réactivité à une nouvelle menace, la simplicité d’utilisation de l’outil, la cible et enfin la pertinence de l’information qui est remontée.\\

\small
\begin{tabular}{|p{2.5cm}|p{3.5cm}|p{3.5cm}|p{2cm}|p{3.5cm}|}
  \hline
  \rowcolor{Gainsboro}Projet & Réactivité & Simplicité & Cible & Pertinence de l'information \\
\hline

  Nessus & Plutôt longue. Réactivité éditeur + temps de scan & Les scans classiques sont simplistes à utiliser mais la personnalisation de ceux-ci est plus compliqué & Entreprises. & Bonne. Seules les vulnérabilités trouvées sont remontées. \\
\hline

  Shalvik Protect & Réactivité égale à celle des éditeurs de programmes. & Très simple il scan et installe les patchs tout seul. & Entreprises. & Bonne. Elle concerne seulement les programmes installés sur la machine. \\
\hline

  CERT/CSIRT & Très bonne. L’équipe est prête à répondre en cas d’urgence. & Très simple pour l’entreprise, ce n'est pas elle qui s’en occupe. & Entreprises, organisations. & Bonne. Elle est adaptée à la cible.\\
\hline

  Cisco Security Manager & Bonne, dès qu’un événement interne est détecté. & Difficile. Nécessite une connaissance des technologies réseaux. & Entreprises, organisations & Bonne, elle concerne des événements internes.\\
\hline

  CERT-XMCO & Bonne. Alerte dès qu’un bulletin est publié. & Assez simple. & Entreprises, organisations. & Mauvaise. Touts les bulletins sont affichés, et le filtrage ne se fait pas forcément très bien. \\
\hline

  Vigilate & Très bonne. Dès qu’une vulnérabilité est rendu publique. & Très simple. Ergonomique. & Entreprises, organisations, revendeurs, webmasters, particuliers & Très bonne. Seule l’information qui concerne l’utilisateur est remontée. \\
\hline
\end{tabular}
\normalsize  

\colorlet{helpful}{lime!70}
\colorlet{harmful}{red!30}
\colorlet{internal}{yellow!20}
\colorlet{external}{cyan!30}
\colorlet{S}{helpful!60!internal}
\colorlet{W}{harmful!50!internal}
\colorlet{O}{helpful!50!external}
\colorlet{T}{harmful!50!external}

\newcommand{\texta}{Positive\\ \tiny (pour atteindre l'objectif)\par}
\newcommand{\textb}{Négative\\ \tiny (pour atteindre l'objectif)\par}
\newcommand{\textcn}{Origine interne\\ \tiny (organisationnelle)\par}
\newcommand{\textdn}{Origine Externe\\ \tiny (environnement)\par}

\newcommand{\back}[1]{\fontsize{60}{70}\selectfont #1}

\begin{tikzpicture}[
  any/.style={minimum width=4cm,minimum height=4cm,%
    text width=3cm,align=center,outer sep=0pt},
  header/.style={any,minimum height=1cm,fill=black!10},
  leftcol/.style={header,rotate=90},
  mycolor/.style={fill=#1, text=#1!75!black}
  ]

  \matrix (SWOT) [matrix of nodes,nodes={any,anchor=center},%
  column sep=-\pgflinewidth,%
  row sep=-\pgflinewidth,%
  row 1/.style={nodes=header},%
  column 1/.style={nodes=leftcol},
  inner sep=0pt]
  {
    &|[fill=helpful]| {\texta} & |[fill=harmful]| {\textb} \\
    |[fill=internal]| {\textcn} & |[mycolor=S]| \back{S} & |[mycolor=W]| \back{W} \\
    |[fill=external]| {\textdn} & |[mycolor=O]| \back{O} & |[mycolor=T]| \back{T} \\
  };

  \node[any, anchor=center] at (SWOT-2-2) {$\bullet$ Informations pertinentes\\ $\bullet$ Personnalisation\\ $\bullet$ Rapidité d'information};
  \node[any, anchor=center] at (SWOT-2-3) {$\bullet$ Pas de scanner de vulnérabilité};
  \node[any, anchor=center] at (SWOT-3-2) {$\bullet$ Besoin de plus de sécurité};
  \node[any, anchor=center] at (SWOT-3-3) {$\bullet$ Beaucoup d'outils existent déjà};

  \end{tikzpicture}
