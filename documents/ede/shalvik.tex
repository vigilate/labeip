\section{Shalvik Protect}
\thispagestyle{plain}
\subsection{Présentation}
Shalvik Protect est une solution de gestion de patch (patch management en anglais).\\
Cette solution permet de gérer les patchs pour les systèmes d’exploitation, les environnements virtuels et les applications tierces.\\

\subsection{Historique}
Originalement appelé HFNetChk (HotFix Network Check), Shalvik protect est créé au début des années 2000.\\
La fonctionnalité de gestion de patch est ajoutée à la version 3 de HFNetChk qui ne sera jamais mis en vente, cette fonctionnalité est donc disponible pour les clients de Shalvik à partir de la version 4.0.\\
Aujourd’hui Shalvik Protect en est à la version 9.\\

\subsection{Description}
Il s'agit d'une solution de gestion de patchs permettant de faciliter l’installation des correctifs.\\
En effet Shalvik Protect va, régulièrement récupérer les nouveaux patchs disponibles et vérifier s'ils sont appliqués sur les machines du réseau dans lequel il est installé.\\
Les patchs  non installés sont signalés à l’utilisateur et Shalvik protect propose ensuite de les installer sur toutes les machines concernées.\\


\subsection{Points positifs}
Les solutions de gestion de patchs permettent de se protéger d’une vulnérabilité dès qu’elle est corrigée.\\

\subsection{Points négatifs}
Les solutions de gestion de patchs ne permettent pas d’être averties de la découverte d’une nouvelle vulnérabilité.\\

Ces solutions ne sont pas non plus exhaustives, en effet elles ne couvrent qu'un certain nombre d’applications.\\

\subsection{Références}
\small
\noindent
\url{http://www.shavlik.com/products/protect/}\newline
\url{http://www.landesk.com/blog/landesk-acquires-shavlik-from-vmware/}\newline
\url{http://windowsitpro.com/systems-management/hfnetchk-microsofts-new-hotfix-tool}\newline
\normalsize
