\section{Shalvik Protect}
\thispagestyle{plain}
\subsection{Présentation}
Shalvik Protect est une solution de gestion de patch (patch management en anglais).\\
Cette solution permet de géré les patch pour les système d’exploitation, les environnements virtuels et les applications tierces.\\

\subsection{Historique}
Originalement appelé HFNetChk (HotFix Network Check), Shalvik protect est créé au début des années 2000.\\
La fonctionnalité de gestion de patch est ajouté à la version 3 de HFNetChk qui ne serait jamais mis en vente, cette fonctionnalité est donc disponible pour les client de Shalvik à partir de la version 4.0.\\
Aujourd’hui Shalvik Protect en est à la version 9.\\

\subsection{Description}
Une solution de gestion de patch permet de facilité l’installation des correctifs.\\

En effet Shalvik Protect vas, régulièrement récupérer les nouveaux patchs disponible et regarder si ils sont appliqués sur les machines du réseaux dans lequel il est insatllé.\\

Les patchs  non installés sont signalés à l’utilisateur et Shalvik protect propose ensuite de les installer sur toutes les machines concernées.\\


\subsection{Points positifs}
Les solutions de gestion de patch permettent de se protéger d’une vulnérabilité dès qu’elle est corrigé.\\

\subsection{Points Négatifs}
Les solutions de gestions de patch ne permettent pas d’être avertit de la découverte d’une nouvelle vulnérabilité.\\

Ces solutions ne sont pas non plus exhaustives, en effet elles ne couvrent que un certains nombre d’applications.\\

\subsection{Références}
\small
\noindent
\url{http://www.shavlik.com/products/protect/}\newline
\url{http://www.landesk.com/blog/landesk-acquires-shavlik-from-vmware/}\newline
\url{http://windowsitpro.com/systems-management/hfnetchk-microsofts-new-hotfix-tool}\newline
\normalsize