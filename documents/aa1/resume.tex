Ce document commence par un bref rappel de notre EIP, Vigilate, un outil de sécurité informatique qui permet de se tenir informé des dernières vulnérabilités publiques rapidement sur tous les programmes que l’utilisateur souhaite vérifier. On mettra en avant les contraintes fonctionnelles et les contraintes non-fonctionnelles comme le fait que notre projet doit être sécurisé, et portable, ou qu'il doit être simple à utiliser et réactif. Les parties vue globale et vue logique découpe le projet pour l'expliquer. Un diagramme d'usecase liste les différentes actions possibles (inscription/ajout de programme/lancer un scan/...) et un diagramme de composants liste les composants en interaction dans note outil (navigateur de l'utilisateur/frontend/backend/vm/flux rss/...). Une partie implémentation décris de façon globale les différentes couches d'implémentation (architecture trois tiers (base de données/application/Interface) ainsi que le model MVT (modele-vue-template) de django.