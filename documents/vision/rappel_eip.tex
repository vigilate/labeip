\section{Qu'est-ce qu’un EIP et Epitech}
Epitech, école de l'innovation et de l'expertise informatique propose un cursus en 5 années, basé sur une pédagogie par projet (à réaliser seul ou en groupe). L'un de ces projets, l’EIP (Epitech Innovating Project), réalisé par groupes de 6 à 15 étudiants, démarre au cours de la 3\ieme{} année et se déroule sur 2 ans et demi. Ce projet doit être particulièrement innovant, car son objectif est d’être commercialisable à la fin de la 5\ieme{} année d’Epitech.

\section{Contexte et périmètre du projet}
Vigilate est un outil qui permet aux entreprises d'être informé instantanément dès qu'une nouvelle vulnérabilité est sortie. Vigilate est un site web qui comprend une partie profil: l'utilisateur aura alors la possibilité de d'ajouter/modifier/supprimer les programmes de son système d'information sur notre base de données. Pour une meilleure synchronisation des programmes et de leurs versions, Vigilate fourni un scanner de programmes. Grâce à la personnalisation des alertes, l'utilisateur pourra définir ses besoins. Il y a en premier lieu le choix de criticité des alertes en fonction des programmes (faible, modéré, urgent) puis le style d'alerte que l'utilisateur préfère: sms, mail, notification via le site internet. Les entreprises voulant encore plus de sécurité pourront avoir une machine virtuelle, elles pourront alors accéder à notre outil tout en gardant leurs données en local.


\section{Glossaire}
\noindent

\vskip 0.1cm
\textbf{- A -}\\
\vskip 0.1cm

\textit{API (Application Programming Interface)}: un enssemble de classes, de méthodes ou de fonction qui sert de façade par laquelle un logiciel offre des services à d'autres logiciels.\\

\vskip 0.1cm
\textbf{- B -}\\
\vskip 0.1cm

\textit{Backend}: un back-end est un terme désignant un étage de sortie d'un logiciel devant produire un résultat. On l'oppose au front-end (aussi appelé un frontal) qui lui est la partie visible de l'iceberg.\\

\vskip 0.1cm
\textbf{- F -}\\
\vskip 0.1cm

\textit{Frontend}: La partie frontend sera la partie avec laquelle l'utilisateur va interagir. Le serveur frontal intercepte les requêtes utilisateur et les ré-envoie vers le serveur backend. \\

\vskip 0.1cm
\textbf{- V -}\\
\vskip 0.1cm

\textit{VM (Machine Virtuelle)}: une machine virtuelle est une illusion d'un appareil informatique créée par un logiciel d'émulation. Le logiciel d'émulation simule la présence de ressources matérielles et logicielles permettant d'exécuter des programmes dans les mêmes conditions que celles de la machine simulée.\\

