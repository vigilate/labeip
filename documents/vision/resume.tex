Vision est un document qui va d'abord faire un rappel du sujet de notre EIP. Vigilate est un outil destiné principalement aux entreprises, il va permettre à celles-ci de se tenir informé immédiatement dès qu'une nouvelle vulnérabilité est connue. Pour cela nous avons décrits dans ce document les besoins de notre projet. Nous avons décidé de mettre notre outil sur un site web pour que le client n'ait pas à installer de logiciel, c'est donc un accès plus rapide avec une simple identification puis une page profil avec chaque programme nécessitant une sécurité immédiate. Elles auront aussi le choix de signer un contrat à la main (par courrier ou par rencontre). Le client pourra aussi préférer garder ses informations dans l'entreprise, dans ce cas, nous lui proposerons une machine virtuelle qui permettra d'avoir un outil fonctionnel tout en laissant les informations chez eux. Nous avons donc un projet qui se découpe en 5 modules. Le premier étant le site web avec une interface graphique user-friendly pour encourager les entreprises à utiliser notre outil, c'est le lien direct qu'aura le client avec Vigilate, il pourra choisir en fonction de ses envies, quels programmes il veut ou ne veut pas surveiller, choisir des niveaux de criticité, mais aussi les choix d'alerte qu'il préfère (mail, sms..). Pour aider le client et qu'il n'ait pas chaque programme à entrer à la main, nous leur donnerons accès à un scanner de programmes, c'est un petit logiciel en plus qui tournera sur la machine et fera ressortir tous les programmes installés sur l'ordinateur, ainsi que leur version. Le résultat de se scanner sera visible depuis le site internet, ces informations seront privées et visibles seulement après identification. Comme dit plus haut, certaines entreprises ont besoin de confidentialité et ne peuvent pas partager leur liste de programmes, pour leur permettre d'utiliser Vigilate, nous proposons une machine virtuelle. Pour finir avec les modules visibles par les clients, nous allons installer un système de paiement par paypal avec 3 sortes d'abonnement pour leur permettre un accès rapide à notre outil. Notre dernier module, l'API est la partie cachée au client, c'est ce qui va nous permettre de récupérer les vulnérabilités puis d'identifier chaque client concerné pour leur envoyer l'information. Pour développer notre projet, nous avons commencé à chercher les technologies les plus adaptées, nous nous arrêtons pour le moment à une partie web avec du Django et de l'AngularJS, et une API en python. Nous utiliserons aussi deux bases de données : PostgreSQL, MariaDB.