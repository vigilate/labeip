\textcolor{myBlue}{\chapter{Contexte}}
\section{Hypothèses}
\begin{itemize}
\item Le site web Vigilate sera développé en python sur le framework Django et le front-end utilisera Angular.js, ces technologies sont reconnues et très performantes.\\
\item Le scanner de programmes sera également développé en python pour une portabilité optimale.\\
\item La machine virtuelle proposée aux clients sera basée sur Debian pour plus de stabilité.\\
\item Vigilate proposera à ses clients plusieurs types d'abonnement qui s'adapteront aux besoins de chacun (20 programmes surveillés / programmes illimités / assistance personnalisée).\\
\item Vigilate alertera d'une nouvelle vulnérabilité via une notification sur le site web, un email ou encore un SMS.\\
\end{itemize}
\section{Contraintes}
\begin{itemize}
\item La réactivité sera un des points principaux à respecter afin de ne laisser s’échapper qu’un minimum de temps entre la diffusion d’une CVE et sa transmission aux clients concernés.\\
\item Être sécurisé afin d’éviter impérativement une potentielle fuite d’informations relative aux clients ou à notre infrastructure.\\
\item La portabilité de la solution, en effet, le projet se veut multiplateforme car l’exploitation des failles ne se limite pas à une seule plateforme.\\
\item Proposer une réaction adaptée au client en fonction de la vulnérabilité découverte (C’est-à-dire couper le service concerné, le suspendre, proposer une mise à jour si elle existe).\\
\item La solution ne devra pas consommer beaucoup de mémoire, au cas où elle est amenée à fonctionner sur des environnements réduits ou limités.\\
\item La fréquence de consultation des nouvelles vulnérabilités sera adaptée à chaque infrastructure. En effet, les machines hébergeant la solution ne devront pas voir leur trafic réseau impacté par l’installation de la solution. Le trafic doit rester optimal en toutes circonstances.
\end{itemize}