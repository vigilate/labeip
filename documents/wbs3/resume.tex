Ce document va dabord commencer par un résumé de notre projet: un outil de suivi de vulnérabilité. Ensuite nous évoquerons différentes hypothèses: Développement en python, site web avec django et angular, vm utilisant debian. Différentes contraintes sont énnoncés: Une grande réactivité dès l'aparition d'un nouvelle vulnérabilité, un serveur sécurisé (audit) afin d'éviter toute fuite d'information.\\
Notre outil se présentera sous forme d'un site internet pour une utilisation facile, il faudra donc d'une part s'occuper du design du site et d'autre part du coeur de l'outil en lui-même (le backend). Toutefois, pour les clients ayant besoin d'un niveau de sécurité supérieur, nous allons mettre en place une machine virtuelle. Cela permettra de contenir les informations sensibles au sein de leur réseau informatique.\\
Une quatrième partie sera de créer un scanner de programmes, permettant à l'utilisateur d'envoyer automatiquement la liste de ses services à surveiller, ce qui permettra de tenir notre base de données à jour. Nous serons aussi présent sur les résaux sociaux, et nous aurons une newsletter.\\