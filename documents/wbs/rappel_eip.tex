\section{Qu'est-ce qu’un EIP et Epitech}
Epitech, école de l'innovation et de l'expertise informatique propose un cursus en 5 années, basé sur une pédagogie par projet (à réaliser seul ou en groupe). L'un de ces projets, l’EIP (Epitech Innovating Project), réalisé par groupes de 6 à 15 étudiants, démarre au cours de la 3\ieme{} année et se déroule sur 2 ans et demi. Ce projet doit être particulièrement innovant, car son objectif est d’être commercialisable à la fin de la 5\ieme{} année d’Epitech.

\section{Sujet de votre EIP}
Le but de Vigilate est d’avertir les utilisateurs des services obsolètes ou potentiellement vulnérables affectant en particulier leur infrastructure (sites web, réseau d'entreprises, logiciels), dans le but de les informer des risques techniques encourus et des éventuelles mises à jour ou corrections à appliquer.
Cela sans effectuer de scan de vulnérabilité.
Nous proposons cependant un outil de scan de programme qui envoie la liste sur notre plateforme web.

\section{Principe de base du système futur}
Vigilate sera composé de quatres grands blocs fonctionnels: site web, programme de scan, backend et vm.\\
La partie frontend du site web communiquera via l’api avec la partie backend.\\
Le programme de scan communiquera également via l’api avec le backend pour mettre à jour la liste de logiciel que veut surveiller l’utilisateur, l'objectif étant de l’informer le plus rapidement possible et d’être simple d’utilisation.\\
La machine virtuelle sera une sorte de copie locale de ce qu’il y a sur nos serveurs. Elle permettra aux entreprises qui le souhaitent d’avoir un niveau de sécurité encore plus important.\\
Cet outil est adapté à plusieurs types de clients: Aussi bien un webmaster souhaitant garder un oeil sur le CMS de son site web, qu’une entreprise voulant veiller à la sûreté de ses services informatiques.\\

\section{Glossaire}
\noindent

\vskip 0.1cm
\textbf{- A -}\\
\vskip 0.1cm

\textit{API (Application Programming Interface)}: un enssemble de classes, de méthodes ou de fonction qui sert de façade par laquelle un logiciel offre des services à d'autres logiciels.\\

\vskip 0.1cm
\textbf{- B -}\\
\vskip 0.1cm

\textit{Backend}: un back-end est un terme désignant un étage de sortie d'un logiciel devant produire un résultat. On l'oppose au front-end (aussi appelé un frontal) qui lui est la partie visible de l'iceberg.\\

\vskip 0.1cm
\textbf{- C -}\\
\vskip 0.1cm

\textit{CVE (Common Vulnerabilities and Exposures)}: dictionnaire d'informations publiques relatives aux vulnérabilités informatiques. Par métonymie, on emploie souvent le terme CVE à la place de CVE ID (ou identifiant CVE), qui lui désigne le numéro qui renvoie à la fiche descriptive complète de cette vulnérabilité. Exemple: CVE-2013-4343.\\

\vskip 0.1cm
\textbf{- F -}\\
\vskip 0.1cm

\textit{Frontend}: La partie frontend sera la partie avec laquelle l'utilisateur va interagir. Le serveur frontal intercepte les requêtes utilisateur et les ré-envoie vers le serveur backend. \\

\vskip 0.1cm
\textbf{- V -}\\
\vskip 0.1cm

\textit{VM (Machine Virtuelle)}: une machine virtuelle est une illusion d'un appareil informatique créée par un logiciel d'émulation. Le logiciel d'émulation simule la présence de ressources matérielles et logicielles telles que la mémoire, le processeur, le disque dur, voire le système d'exploitation et les pilotes, permettant d'exécuter des programmes dans les mêmes conditions que celles de la machine simulée.\\

