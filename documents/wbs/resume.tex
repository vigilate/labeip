Ce document va d'abord parler de notre projet, un outil pour être au top de la sécurité, puis va expliquer par un schéma et un dictionnaire les différentes parties de Vigilate. Dans un premier temps, nous avons essayé de réfléchir à comment on pourrait faire et aux contraintes qui pouvaient nous toucher. Pour l'aspect technique, nous nous sommes dit que de faire un site web en Django était la solution la plus pratique, car le backend sera en python (langage le plus maîtrisé par les membres concernés de l'équipe). Il va aussi falloir réfléchir au design et à l'ergonomie, car nous voulons un outil simple et rapide à utiliser, sans toutefois oublier le serveur qui se devra d'être le plus sécurisé possible. Notre outil se présentera sous forme d'un site internet pour une utilisation facile, il faudra donc d'une part s'occuper du design du site et d'autre part du coeur de l'outil en lui-même (le backend). Toutefois, pour les clients ayant besoin d'un niveau de sécurité supérieur, nous allons mettre en place une machine virtuelle.\\
Une quatrième partie sera de créer un scanner de programmes, permettant à l'utilisateur d'envoyer automatiquement la liste de ses services à surveiller, ce qui permettra de tenir notre base de données à jour.\\