\section{Tests pendant la phase de développement}
\begin{itemize}
\item Utilisation de l'outil Travis CI fourni avec GitHub afin d'assurer une intégration continue du projet.\\
\end{itemize}
Travis permet d'automatiser une batterie de tests à chaque contribution au projet et permet par la même de rejeter la contribution si elle ne valide pas la série de tests pré-programmés.\\
De cette façon, l'intégration continue grâce à Travis Ci nous permet de s'assurer que chaque modification ne produit pas de régression au sein du développement du projet.\\

 
Travis nous permet donc d'automatiser les tests de régression et d'avoir l'état précis de l'avancement du projet.\\

\section{Tests avant la sortie final}
\begin{itemize}
\item Test d'intrusion complet (white box) garantissant la sécurité de l'ensemble de la solution. Cette analyse en profondeur de la sécurité de notre solution nous permet de garantir la fiabilité et la stabilité de notre projet lorsque celui-ci est confrontée à des attaques diverses et variées. C'est une étape non négligeable dans le développement d'un projet, surtout lorsque le thème abordé est en rapport avec la sécurité applicative.\\
\item Test complet de l'interface utilisateur. Cette étape finale de test est obligatoire dans le sens ou elle nous permet de bien assurer le contrôle qualité de notre produit et évite tout désagrément pour l'utilisateur final.\\
\end{itemize}