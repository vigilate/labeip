Le cahier des charges débute par un bref rappel de notre EIP, Vigilate, un outil de sécurité informatique qui permet de se tenir informé des dernières vulnérabilités publiques rapidement sur tous les programmes que l’utilisateur souhaite vérifier.\\
On mettra en avant les contraintes fonctionnelles et les exigences non fonctionnelles comme le fait que notre projet doit être sécurisé, ou alors la réactivité de notre outil.\\
Notre projet est réalisé par 6 personnes. Le document présente d’une part la structure du projet : un site web, le back-end, une machine virtuelle et un scanneur de programme et d’autre part les différentes fonctionnalités du projet à partir d’un document UML (connexion de l’utilisateur, possibilité de modifier ses préférences : programmes à surveiller, style d’alerte souhaité...).\\
Nous présentons ensuite l’organisation prévue : 2 personnes sur le site web, 2 sur le back-end, 1 sur la machine virtuelle et 1 sur le scanneur de programme.\\
Nous développons ensuite la méthodologie utilisée, basée sur la méthode Agile (Scrum). Pour ce faire nous devons construire un produit backlog avant de commencer à développer le projet, puis organiser un nombre de sprints adéquat, d’une durée relativement courte (principe de la méthode agile) mais qui permettront d’avoir un projet fonctionnel à chaque fin de sprint avec chaque tâche décrite dans un backlog de sprint.\\
Nous présenterons également l’environnement et les outils utilisés (github pour  héberger nos documents...). Nous avons réalisé un template de mise en page pour toute notre documentation avec Latex. Nous avons choisi de respecter la norme Python Pep8 pour le développement.\\
Dans une dernière partie nous proposerons une description des tests, utilisés pendant le développement notamment grâce à l’outil Travis et les tests une fois l’outil terminé.\\