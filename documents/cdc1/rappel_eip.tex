\section{Qu'est-ce qu’un EIP et Epitech}
Epitech, école de l'innovation et de l'expertise informatique propose un cursus en 5 années, basé sur une pédagogie par projet (à réaliser seul ou en groupe). L'un de ces projets, l’EIP (Epitech Innovating Project), réalisé par groupes de 6 à 15 étudiants, démarre au cours de la 3\ieme{} année et se déroule sur 2 ans et demi. Ce projet doit être particulièrement innovant, car son objectif est d’être commercialisable à la fin de la 5\ieme{} année d’Epitech.

\section{Sujet de votre EIP}
Le but de Vigilate est d’avertir les utilisateurs des services obsolètes ou potentiellement vulnérables affectant en particulier leur infrastructure (sites web, réseau d'entreprises, logiciels), dans le but de les informer des risques techniques encourus et des éventuelles mises à jour ou corrections à appliquer.
Cela sans effectuer de scan de vulnérabilité.
Nous proposons cependant un outil de scan de programme qui envoie la liste sur notre plateforme web.

\section{Principe de base du système futur}
Vigilate est un outil permettant de tenir informé chaque utilisateur des dernières vulnérabilités publiques pour chacun de ses programmes. Il est destiné principalement aux entreprises mais aussi aux personnes intéressées par la sécurité de leur système.\\
Un utilisateur s’inscrira sur notre plateforme web, donnera les informations sur les programmes dont il souhaite être informé et le moyen d’alerte qu’il préfère (mail, sms, notifications). L’objectif principal est de se tenir informé vite pour garantir la meilleure sécurité possible dans les systèmes d’information. Un scanneur de programme sera aussi mis en place pour permettre d’être toujours à jour dans nos bases de données (changement de version, installation de nouveaux programmes). Pour encore plus de sécurité nous allons proposer une machine virtuelle. Cela permettra aux utilisateurs ayant un besoin accru de sécurité, de garder les informations sensibles sur leur propre réseau.

L’équipe est composée de 6 personnes:\\
Kévin Soules\\
Morgane Harscoat\\
Daniel Mercier\\
Flavien Keller\\
Manuel Poncet\\

A la fin de EIP, nous espérons avoir un site web fonctionnel qu’on pourra proposer aux entreprises.

\section{Glossaire}
\noindent

\vskip 0.1cm
\textbf{- L -}\\
\vskip 0.1cm

\textit{Lorem}: lorem ipsum dolor sit amet\\

\vskip 0.1cm
\textbf{- I -}\\
\vskip 0.1cm

\textit{Ipsum}: ipsum dolor sit amet\\

