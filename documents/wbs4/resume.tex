Le WBS comporte 3 parties, la présentation de notre EIP. Vigilate est un outil de sécurité informatique qui permet de se tenir informé des dernières vulnérabilités publiques rapidement sur tous les programmes que l’utilisateur souhaite vérifier. La deuxième partie portera sur le contexte (les hypothèses et les contraintes). En effet pour avoir un EIP fonctionnel et répondant au cahier des charges, nous allons devoir nous pencher sur différents problèmes comme le choix des technologies pour une meilleure portabilité (programme de scan) ou une meilleure stabilité (machine virtuelle). Aussi, nous proposons un produit qui se base sur la rapidité et la sécurité. En effet, l’utilisateur de Vigilate devra être averti le plus vite possible à chaque nouvelle vulnérabilité sortie, on va donc devoir mettre cette contrainte en priorité dans notre développement. Pour ce qui est de la sécurité, nous proposons un outil qui permet d’être averti des nouvelles failles, il faut donc que notre produit soit irréprochable à ce niveau-là.\\
La troisième partie de ce document est le WBS, une représentation graphique du découpage de notre travail prévu sur Vigilate. La partie développement avec le site internet, l’api, le programme de scan, la machine virtuelle, mais aussi ce qu’on devra faire pour mettre en avant notre projet avec de communication par réseaux sociaux et la création d’une newsletter. Ce schéma est accompagné d’un dictionnaire qui permet d’être plus précis sur chaque tache ainsi que le pourcentage d’avancement sur chaque tache.