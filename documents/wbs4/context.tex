\textcolor{myBlue}{\chapter{Contexte}}
\section{Hypothèses}
\begin{enumerate}
\item Nous avons choisi de développer le site web Vigilate en python avec le framework Django. Le front-end sera pour sa part réalisé grâce à Angular.js. Ces technologies sont reconnues comme performantes, d'où notre choix. Cependant, ne les ayant jamais utilisées auparavant, nous ne sommes pas certains qu'elles seront parfaitement adaptées à notre projet. Dans ce cas, nous devrons peut-être décider de changer de technologies. Si cela arrive, nous serons confrontés à des temps de développement plus importants pour palier à ces problèmes.\\
\item Nous pensons commercialiser Vigilate sous la forme d'abonnements comprenant plus ou moins de services pour s'adapter à chaque type de structure. Cependant, nous ne sommes pas certains que ce modèle conviendra aux futurs utilisateurs et nous devrons alors imaginer un nouveau modèle de vente en conséquence.\\
\item Le site web Vigilate sera développé en python sur le framework Django et le front-end utilisera Angular.js, ces technologies sont reconnues et très performantes.\\
\item Le scanner de programmes sera également développé en python pour une portabilité optimale.\\
\item La machine virtuelle proposée aux clients sera basée sur Debian pour plus de stabilité.\\
\item Vigilate proposera à ses clients plusieurs types d'abonnement qui s'adapteront aux besoins de chacun (20 programmes surveillés / programmes illimités / assistance personnalisée).\\
\item Vigilate alertera d'une nouvelle vulnérabilité via une notification sur le site web, un email ou encore un SMS.
\end{enumerate}

\section{Contraintes}
\begin{enumerate}
\item La réactivité sera un des points principaux à respecter afin de ne laisser s’échapper qu’un minimum de temps entre la diffusion d’une CVE et sa transmission aux clients concernés.\\
\item Être sécurisé afin d’éviter impérativement une potentielle fuite d’informations relative aux clients ou à notre infrastructure.\\
\item La portabilité de la solution, en effet, le projet se veut multiplateforme, car l’exploitation des failles ne se limite pas à une seule plateforme.\\
\item Proposer une réaction adaptée au client en fonction de la vulnérabilité découverte (C’est-à-dire couper le service concerné, le suspendre, proposer une mise à jour si elle existe).\\
\item La solution ne devra pas consommer beaucoup de mémoire, au cas où elle est amenée à fonctionner sur des environnements réduits ou limités.\\
\item La fréquence de consultation des nouvelles vulnérabilités sera adaptée à chaque infrastructure. En effet, les machines hébergeant la solution ne devront pas voir leur trafic réseau impacté par l’installation de la solution. Le trafic doit rester optimal en toutes circonstances.\\
\end{enumerate}