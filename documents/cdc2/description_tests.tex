\section{Tests pendant la phase de développement}
\begin{itemize}
\item Utilisation de l'outil Travis CI fourni avec GitHub afin d'assurer une intégration continue du projet.\\
\end{itemize}
Travis CI permet d'automatiser une suite de tests à chaque contribution au projet et permet par la même occasion de rejeter la contribution si celle-ci ne valide pas la série de tests préprogrammés qui sont jugés indispensables à la pérennité du produit.\\
De cette façon, l'intégration continue grâce à Travis CI nous permet de s'assurer que chaque modification ne produit pas de régression au sein du développement du projet.
\\
Travis nous permet donc d'automatiser les tests de régression et d'avoir l'état précis de l'avancement du projet simplement et rapidement.\\

\section{Tests avant la sortie final}
\begin{itemize}
\item Un test d'intrusion complet en boîte blanche garantissant la sécurité de l'ensemble de la solution. Cette analyse en profondeur de la sécurité de notre solution nous permet de garantir la fiabilité et la stabilité de celle-ci lorsqu’elle est confrontée à des attaques diverses et variées. C'est une étape importante dans le développement d'un projet, à plus forte raison lorsque la cible produit est en rapport avec la sécurité applicative.\\
\item Un test complet de l'interface utilisateur, cette étape finale de test est essentielle car elle nous permet de bien assurer le contrôle qualité du produit et évite tout désagrément pour l'utilisateur final.\\
\end{itemize}