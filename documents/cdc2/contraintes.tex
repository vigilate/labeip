\section{Contraintes fonctionnelles}
\begin{itemize}
\item Portabilité\\
La portabilité sera la principale contrainte fonctionnelle à laquelle la réalisation du projet devra se plier. La machine virtuelle devra être conçue de façon à pouvoir être utilisable sur les systèmes d'exploitation les plus répandus, il en va de même pour le site web, devant impérativement prendre en charge une grande majorités de navigateurs, à jour ou non, afin que son fonctionnement ne soit pas altéré selon la configuration de l’utilisateur.
\end{itemize}

\section{Exigences non fonctionnelles}
\begin{itemize}
\item Simplicité de l'interface\\
Dans un souci de simplicité d'utilisation, il faudra que l'interface utilisateur de notre système soit intuitive et à la portée de tous, de façon à favoriser la prise en main de cette solution par un utilisateur lambda, n'ayant aucune compétence avancée en informatique.\\
Différents types d'alertes pourront être envoyés au client, une alerte détaillée concernant la vulnérabilité, ses détails techniques et ses caractéristiques, pouvant être destinée à l'équipe informatique du client, et une alerte plus concise pouvant par exemple être destinée à un responsable.

\item Réactivité\\
Afin d'optimiser l'efficacité de notre solution, la réactivité est l'un des points clés à respecter.\\
Dans le but d'informer les utilisateurs des vulnérabilités les concernant dans les plus brefs délais, nous nous devrons de concevoir et maintenir un système réactif au niveau la récupération et la transmission des données, cela permettant au client de faire corriger les vulnérabilités de son système au plus vite.\\

\item Concevoir un système et une plateforme sécurisés\\
Le but de notre solution étant de favoriser la sûreté des services utilisés par nos clients, avoir un produit et une infrastructure stables fait également partie des points clés à respecter. Il sera donc impératif de sécuriser correctement le site, les différents services que nous utilisons et par conséquent le stockage de nos données, afin d'éviter toutes fuites de données relatives à un client, à son système informatique, ou à notre propre infrastructure. Enfin, le système en machine virtuelle destiné à être installé chez le client devra être développé de façon à être imperméable à toute forme d'attaque actuellement prévisible, il va de soi qu'après sa fonction principale de détection et d'alerte, la priorité de ce système concerne le fait qu'il ne puisse, ou le moins possible, être lui-même compromis par une attaque informatique.
  
\end{itemize}