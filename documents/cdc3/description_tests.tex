\section{Tests pendant la phase de développement}
L’utilisation de l'outil \href{https://travis-ci.org/}{Travis-CI} fourni avec \href{http://fr.wikipedia.org/wiki/GitHub}{GitHub} afin d'assurer une intégration continue du projet.\\
Travis CI permet d'automatiser une suite de tests à chaque contribution au projet et permet par la même occasion de rejeter la contribution si celle-ci ne valide pas la série de tests préprogrammés qui sont jugés indispensables à la pérennité du produit.\\
De cette façon, l'intégration continue grâce à Travis CI nous permet de s'assurer que chaque modification ne produit pas de régression au sein du développement du projet.\\
Travis nous permet donc d'automatiser les tests de régression et d'avoir l'état précis de l'avancement du projet simplement et rapidement.\\
A noter que Travis couvre donc toute la partie tests unitaires et tests fonctionnels nécessaires au développement de notre projet.\\

\section{Tests de validation}
\begin{itemize}
\item Validation fonctionnelle\\
Ces tests auront pour objectif la vérification du respect des exigences imposées à la création du projet. Vérifier si les besoins et les réponses à ces besoins sont respectés et conformes aux attentes du client.\\
\item Validation solution\\
Ces tests auront pour objectif la vérification des exigences d’utilisations de la solution, chaque cas d’utilisation du projet sera donc testé individuellement et validé individuellement, de façon à assurer une couverture maximale et précise de la validation de l’ensemble de la solution.\\
\item Validation performance\\
Ces tests auront pour objectif la vérification de la conformité de l’ensemble du projet face à des utilisations intenses afin d’essayer de mettre en évidence des problèmes liés à la stabilité et à la robustesse de la solution (Non résistance à un pic de charge par exemple).\\
\item Test de vulnérabilité\\
Des tests d'intrusions complet en \href{http://fr.wikipedia.org/wiki/Bo\%C3\%AEte_blanche}{boîte blanche} garantissant la sécurité de l'ensemble de la solution.
Cette analyse en profondeur de la sécurité de notre solution nous permet de garantir la fiabilité et la stabilité de celle-ci lorsqu’elle est confrontée à des attaques diverses et variées.\\
C'est une étape importante dans le développement d'un projet, à plus forte raison lorsque la cible produit est en rapport avec la sécurité applicative. C'est pour cette raison qu'un test d'intrusion sera réalisé tout les 3 mois.\\

\end{itemize}