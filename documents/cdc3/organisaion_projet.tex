\section{Méthode de travail}
\begin{itemize}
\item Réunion internes hebdomadaire : Ces réunions nous permettront de nous tenir informés de l’avancement de chacun malgré le décalage horaire et la distance géographique. Durant ces réunions nous parlerons du travail effectué durant la dernière semaine, évoqueront les difficultés que nous avons pu rencontrer afin de nous entraider, cela nous permettra également de donner notre avis sur le travail de chaque membre du groupe et de maintenir une cohésion et une bonne dynamique de travail. Ces réunions nous serviront aussi à préparer les suivis mensuels avec le Lab EIP.\\
\item Méthode Scrum (agile) : Méthodologie qui consiste à créer un product backlog avant le développement en définissant toutes les fonctionnalités du projet avec un niveau de priorité, de fonctionner avec un système de sprints c’est à dire avec des phases de développement assez courtes pour avoir un projet fonctionnel, en suivant un sprint backlog (un liste de tâches précise définit avant chaque sprint). Outil utilisé: redmine.\\
\end{itemize}
En premier temps, il nous faudra définir une durée pour chaque sprint, comme nous ne serons pas tous dans les mêmes pays, nous ferons sûrement des sprints un peu plus longs (environ 15 jours). Avant chaque début de sprint, nous ferons une réunion pour définir toutes les tâches à réaliser, le plus précisément possible. Nous déterminerons le temps estimé ainsi que les attributions aux membres du groupe. Nous ajouterons toutes ces informations sur le Redmine, qui sera notre planning commun et nous pourrons voir l’avancée de chaque membre du groupe au fur et à mesure du sprint. Nous devrons aussi à chaque fois qu’une tache et finit la mettre sur le git sur une branche commune pour ne pas avoir de problème de version. À la fin de la période définie, nous ferons de nouveau une réunion, à partir de la branche principale, nous testerons toutes les nouvelles fonctionnalités en notant ce qui pourrait être mieux, être modifié. Si ce sont de courte modification, nous nous laisserons quelques jours pour corriger puis refaire une réunion pour définir le sprint suivant. En cas de retard ou de grosse correction, les tâches du sprint précédent seront reporter sur le suivant. Comme nous serons 3 et 2 dans la même ville, il nous sera plus facile de communiquer et se voir.\\
\section{Outils de travail}
\begin{itemize}
\item Git : grâce à Git nous pourrons constater de l’avancement de chaque membre et consulter le travail des autres de manière simplifiée. Nous pouvons ainsi nous entraider et accéder à la dernière version du projet sans démarche spécifique de notre part, tout est automatiquement mis à jour pour tous les membre sur le répertoire.\\
\item Hangout : nous utilisons une discussion Hangout afin de pouvoir communiquer facilement en interne malgré le décalage horaire entre les membres. Nous utilisons également la fonctionnalité Appel de Hangout pour nos réunions hebdomadaires.\\
\item Travis : à chaque commit sur github, Travis exécute une batterie de tests afin de vérifier que le projet est toujours fonctionnel et que des bugs n’ont pas été ajoutés.\\
\item Redmine : cet outil de gestion de projet nous permettra de bien appliquer les méthodes agiles, faciliter notre travail durant le développement de Vigilate ainsi que de respecter nos jalons.\\
\end{itemize}
\section{Planning}
\begin{itemize}
\item 09/15 : Développement de Vigilate\\
\item 01/16 : Communication passive (page facebook, site vitrine, twitter, newsletter...)\\
\item 09/16 : Bêta test interne du projet\\
\item 09/16 : Lancement d’actions de communication\\
\item 12/16 : Produit fini\\
\item 01/17 : Tests finaux\\
\item 01/17 : Soutenances finales\\
\end{itemize}


Semaine type de travail :\\
\begin{itemize}
\item 1 journée dédiée au travail sur l’EIP\\
\item 1 réunion interne\\
\end{itemize}

\section{Répartition du travail}
La réalisation de Vigilate sera découpée en “grandes parties” qui seront assignées à chaque membre du groupe de la manière suivante. (Bien sûr il n’est pas exclu qu’un membre intervienne sur la partie de quelqu’un d’autre de manière ponctuelle en cas de besoin.)\\
\begin{itemize}
\item Site web (2 personnes)\\
\item Programme de scan (1 personne)\\
\item Backend / API (2 personnes)\\
\item Machine virtuelle (1 personne)\\
\end{itemize}